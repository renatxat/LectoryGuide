%%%%%%%%%%%%%%%%%%%%%%%%%%%%%%%%%%%%%%%%%%%%%%%%%%%%%%%%%%%
%%%% Page format
%%%%%%%%%%%%%%%%%%%%%%%%%%%%%%%%%%%%%%%%%%%%%%%%%%%%%%%%%%%

% \usepackage[14pt]{extsizes}         % enables more font size (use extarticle instead)
\usepackage{geometry}               % geometry of page
\usepackage{setspace}               % spacing between paragraphs
% \usepackage{multicol}               % for text in several columns (for articles)
% \usepackage{soulutf8}               % Модификаторы начертания
\usepackage{indentfirst}            % make indent of first paragraph in section
\usepackage[strict]{changepage}     % for 'adjustwidth' environment
% \usepackage{titleps}                % колонтитулы
\usepackage[pagestyles]{titlesec}   % sections' parameters
%%% More:
% sections: https://ru.overleaf.com/learn/latex/Sections_and_chapters#Customize_chapters_and_sections
% titleformat{} and titleformat*{}: https://tex.stackexchange.com/a/476727/329405
% Invalid margin is the bug of 'titlesec' package. [frame] solves the problem but has a strange design
\titleformat*{\section}{\LARGE\bfseries}
% \titleformat{\section}{\LARGE\bfseries}{Section~\thesection.}{.2em}{}
\titleformat*{\subsection}{\Large\bfseries}
\titleformat*{\subsubsection}{\large\bfseries}
% \titleformat{\subsubsection}[frame]{\large\bfseries}{\S~\thesubsubsection~}{.4em}{}   % extraordinary design
% \titleformat{\subsubsection}{\large\bfseries}{\S~\thesubsubsection~}{.4em}{}
% \titlelabel{\thetitle.\quad}        % text before title of section/subsection/...

% \usepackage{titleps}                % колонтитулы, не совместим с пакетом "titlesec"
\usepackage[normalem]{ulem}
\usepackage{interval}

\usepackage{lscape}     % for "\begin{landscape}" (landscape page)
\usepackage{rotating}   % rotating text, pictures and tables on the page (e.g. "\begin{sideways}", "\begin{turn}{45}", "\begin{sidewaysfigure}[htbp]", "\begin{sidewaystable}[htpb]")
%%% More: https://latex-tutorial.com/landscape-page/

\usepackage[perpage]{footmisc}  % to reset footnote counter on each page.

\usepackage{xargs}      % use more than one optional parameter in new commands
\usepackage[colorinlistoftodos,prependcaption,textsize=normalsize]{todonotes}     % for comments
% \usepackage{calc}       % for package "todonotes"

\usepackage{lipsum} % dummy texts for test. Usage: \lipsum[1-4] (prints 4 paragraphs)

%%%%%%%%%%%%%%%%%%%%%%%%%%%%%%%%%%%%%%%%%%%%%%%%%%%%%%%%%%%
%%%% Counters
%%%%%%%%%%%%%%%%%%%%%%%%%%%%%%%%%%%%%%%%%%%%%%%%%%%%%%%%%%%

%%% The sectioning levels have the following numbers:
%%% -1 part
%%% 0 chapter % in article it is 'part'
%%% 1 section
%%% 2 subsection
%%% 3 subsubsection
%%% 4 paragraph
%%% 5 subparagraph

%%%% toc - table of content
% \setcounter{tocdepth}{2} % + subsections
% \setcounter{tocdepth}{3} % + subsubsections
\setcounter{tocdepth}{4} % + paragraphs
% \setcounter{tocdepth}{5} % + subparagraphs


\setcounter{secnumdepth}{3}     % sec - sections. {0} -- disable section numeration
% \setcounter{section}{-1}        % numeration from 0

%%%%%%%%%%%%%%%%%%%%%%%%%%%%%%%%%%%%%%%%%%%%%%%%%%%%%%%%%%%
%%%% Language of document
%%%%%%%%%%%%%%%%%%%%%%%%%%%%%%%%%%%%%%%%%%%%%%%%%%%%%%%%%%%

\usepackage{iftex}            % \iftutex
\usepackage{cmap}					    % enable search in PDF


% Handle special symbols. May be undefined when LuaTex is used.
\iftutex % to detect if Unicode-aware engine (LuaTeX or XeTeX) is used

\usepackage{fontspec}
\usepackage{unicode-math} % loads 'fontspec' automatically
% \setmainfont{Noto Serif}
% \setsansfont{Noto Sans}
% \setmonofont{Noto Sans Mono}

\usepackage[babelshorthands=true]{polyglossia}
\let\lang\relax\let\endlang\relax % avoid conflict with complexity.sty

\setdefaultlanguage{russian}
\setotherlanguage{english}

\defaultfontfeatures{Ligatures=TeX}

\setmainfont{Times New Roman}
\setsansfont{Arial}
\setmonofont[Contextuals=Alternate]{Fira Code}

\newfontfamily\cyrillicfont{Times New Roman}
\newfontfamily\cyrillicfontsf{Arial}
\newfontfamily\cyrillicfonttt[Contextuals=Alternate]{Fira Code}

\newfontfamily\englishfont{Times New Roman}
\newfontfamily\englishfontsf{Arial}
\newfontfamily\englishfonttt[Contextuals=Alternate]{Fira Code}

\else

\usepackage[T2A]{fontenc}			    % encoding of pdf (old 8-bit technologies)
\input{glyphtounicode}
\input{glyphtounicode-cmr} %from pdfx package
\pdfgentounicode=1

\fi

\usepackage[utf8]{inputenc}             % encoding of sources (.tex files)
\usepackage{csquotes}                   % for bable + bibletex (\enquote + \textquote)
\usepackage{hyphsubst}                % suppress "No hyphenation patterns were preloaded for"
\usepackage[english, russian]{babel}	% localisation + word breaks
\usepackage{tempora} % this supports Cyrillic for russian
% \frenchspacing                          % enables equal spacing between sentences and words
% \usepackage{courier}                    % modern font family

%%% Hyphenation: https://tex.stackexchange.com/questions/550978/hyphenrules-no-longer-defined-in-polyglossia
% \usepackage{polyglossia}

%%%%%%%%%%%%%%%%%%%%%%%%%%%%%%%%%%%%%%%%%%%%%%%%%%%%%%%%%%%
%%%% Table of content
%%%%%%%%%%%%%%%%%%%%%%%%%%%%%%%%%%%%%%%%%%%%%%%%%%%%%%%%%%%

\usepackage{tocloft}
% \usepackage[titletoc]{appendix}
% \usepackage[dotinlabels]{titletoc}

% \renewcommand{\contentsname}{Оглавление}
% \renewcommand{\appendixtocname}{Приложения}

% \renewcommand{\cftpartleader}{\cftdotfill{\cftdotsep}}

%\renewcommand{\cftchapleader}{\cftdotfill{\cftdotsep}}

\renewcommand{\cftsecleader}{\cftdotfill{\cftdotsep}}
% \renewcommand{\appendixtocname}{Приложения}

%%%%%%%%%%%%%%%%%%%%%%%%%%%%%%%%%%%%%%%%%%%%%%%%%%%%%%%%%%%
%%%% Enumerate and itemize
%%%%%%%%%%%%%%%%%%%%%%%%%%%%%%%%%%%%%%%%%%%%%%%%%%%%%%%%%%%

\usepackage{enumerate}
\usepackage{enumitem}   % управляет тем, как выглядит нумерованный список (внутри окружения enumerate можно использовать счётчики enumi, enumii, enumiii, ...)
\usepackage{pifont}     % more styles for bullets

%% Settings are in Custom.tex

% %%% Enumerate
% \setenumerate[1]{label=\arabic*), fullwidth, itemindent=\labelsep + \labelwidth + \leftmargin,
%   listparindent=\parindent,  leftmargin=0.5\parindent, topsep=0pt}
% \setenumerate[2]{label=\arabic{enumi}.\arabic*), fullwidth, itemindent=0.5\parindent,
%   listparindent=\parindent, leftmargin=0.5\parindent, topsep=0pt}
% \setenumerate[3]{label=\Alph*), fullwidth, itemindent=0.5\parindent,
%   listparindent=\parindent, leftmargin=0.5\parindent, topsep=0pt}
% \setenumerate[4]{label=\Roman*), fullwidth, itemindent=0.5\parindent,
%   listparindent=\parindent, leftmargin=0.5\parindent, topsep=0pt}

% %%% Itemize
% %%% More items: https://latex-tutorial.com/bullet-styles/
% \setitemize[1]{label=$\blacksquare$, fullwidth, itemindent=0.5\parindent,
%   listparindent=\parindent,  leftmargin=0.5\parindent, topsep=0pt}
% \setitemize[2]{label=\ding{117}, fullwidth, itemindent=0.5\parindent,
%   listparindent=\parindent, leftmargin=0.5\parindent, topsep=0pt}
% \setitemize[3]{label=$\bullet$, fullwidth, itemindent=0.5\parindent,
%   listparindent=\parindent, leftmargin=0.5\parindent, topsep=0pt}
% \setitemize[4]{label=*, fullwidth, itemindent=0.5\parindent,
%   listparindent=\parindent, leftmargin=0.5\parindent, topsep=0pt}

%%%%%%%%%%%%%%%%%%%%%%%%%%%%%%%%%%%%%%%%%%%%%%%%%%%%%%%%%%%
%%%% Tables
%%%%%%%%%%%%%%%%%%%%%%%%%%%%%%%%%%%%%%%%%%%%%%%%%%%%%%%%%%%

\usepackage{array,tabularx,tabulary,booktabs} % tables
\usepackage{longtable}                        % long tables
\usepackage{multirow}                         % merging rows inside tables

%%%%%%%%%%%%%%%%%%%%%%%%%%%%%%%%%%%%%%%%%%%%%%%%%%%%%%%%%%%
%%%% Math
%%%%%%%%%%%%%%%%%%%%%%%%%%%%%%%%%%%%%%%%%%%%%%%%%%%%%%%%%%%

\usepackage[intlimits]{amsmath}
\usepackage{amsthm}             % for theorems' environments
\usepackage{amsfonts}
\usepackage{amssymb}
\usepackage{mathrsfs}           % example: $\mathscr F$
\usepackage{mathtools}
% \usepackage[usenames]{color}  % \usepackage[table]{xcolor}
\usepackage{xcolor}
\usepackage{esint}
\usepackage{bm}                 % \bm{} for bold math symbols
\usepackage{bbm}
\usepackage{dsfont}
\usepackage{seqsplit}           % for line wrap ups
\usepackage{makecell}
\usepackage{{./Headings/Style/slashbox}}
\usepackage{xfrac}
\usepackage{easybmat}           % block matrices (\BMAT).
\usepackage{accents}            % math accents (\star{}, \diamond{}, \undertilde{}, ...)

\numberwithin{equation}{section}        % Нумерация вида (номер_секции).(номер_уравнения)
% \counterwithin{equation}{section}       % ????????????????
\mathtoolsset{showonlyrefs=false}       % Номера только у формул с \eqref{} в тексте.

%%%%%%%%%%%%%%%%%%%%%%%%%%%%%%%%%%%%%%%%%%%%%%%%%%%%%%%%%%%
%%%% References
%%%%%%%%%%%%%%%%%%%%%%%%%%%%%%%%%%%%%%%%%%%%%%%%%%%%%%%%%%%

\usepackage{varioref} % vref
\usepackage{hyperref}
\usepackage{cleveref} % cref

\usepackage{url}
\makeatletter
\g@addto@macro{\UrlBreaks}{\UrlOrds}
\makeatother

%%%%%%%%%%%%%%%%%%%%%%%%%%%%%%%%%%%%%%%%%%%%%%%%%%%%%%%%%%%
%%%% Bibliography
%%%%%%%%%%%%%%%%%%%%%%%%%%%%%%%%%%%%%%%%%%%%%%%%%%%%%%%%%%%

\usepackage[backend=biber]{biblatex}

%%%%%%%%%%%%%%%%%%%%%%%%%%%%%%%%%%%%%%%%%%%%%%%%%%%%%%%%%%%
%%%% Pictures and images
%%%%%%%%%%%%%%%%%%%%%%%%%%%%%%%%%%%%%%%%%%%%%%%%%%%%%%%%%%%

\usepackage{graphicx, epsfig}     % images
% \usepackage[final]{graphicx, epsfig}
\graphicspath{{./Images/}} % relative (to main.tex) path of images
% \graphicspath{Images/} -- relative to file containing the \includegraphics command
\usepackage{pgfplots}
\usetikzlibrary{arrows}
\usetikzlibrary{arrows.meta}
\pgfplotsset{width=10cm,compat=newest}
\pgfkeys{/pgf/trig format=rad}

\usepackage{subcaption}
\usepackage{caption}
% \captionsetup[figure]{labelsep = period}

% \usepackage{watermark}
%\thiswatermark{\put(-100,-550){\includegraphics[scale=0.8]{images/guk.jpg}} }

\usepackage[most]{tcolorbox}
\definecolor{block-gray}{gray}{0.85}
\newtcolorbox{myquote}{colback=block-gray,grow to right by=-0mm,grow to left by=-0mm,
  boxrule=0pt,boxsep=0pt,breakable}

\usepackage{wrapfig}            % для плавающих объектов (изображений, таблиц)
\usepackage[export]{adjustbox}  % "fbox" and "frame" -- border around the image
\setlength\fboxsep{3pt}         % Отступ рамки \fbox{} от рисунка
\setlength\fboxrule{1pt}        % Толщина линий рамки \fbox{}

% param to make pictures with width==\textwidth
\newlength{\fboxaddlen}
\setlength{\fboxaddlen}{2\fboxsep + 2\fboxrule}

%%% Графика
\usepackage{tikz}        % Графический пакет tikz
% \usepackage{tikz-cd}     % Коммутативные диаграммы
% \usepackage{tkz-euclide} % Геометрия
% \usepackage{stackengine} % Многострочные тексты в картинках
% \usetikzlibrary{angles, babel, quotes}
% \usetikzlibrary{backgrounds,automata}

% https://www.overleaf.com/learn/latex/LaTeX_Graphics_using_TikZ%3A_A_Tutorial_for_Beginners_(Part_3)%E2%80%94Creating_Flowcharts
% Also see "tikzpicture" example in "Custom.tex"
\usetikzlibrary{shapes.geometric, arrows}
\tikzstyle{rect} = [rectangle,
rounded corners,
minimum width=3cm,
minimum height=1cm,
text centered,
draw=black,
fill=white]
\tikzstyle{arrow} = [thick,->,>=stealth]

\usepackage{circuitikz} % online graphical editor tikzmaker: https://tikzmaker.com/editor

%%%%%%%%%%%%%%%%%%%%%%%%%%%%%%%%%%%%%%%%%%%%%%%%%%%%%%%%%%%
%%%% Algorithms and code
%%%%%%%%%%%%%%%%%%%%%%%%%%%%%%%%%%%%%%%%%%%%%%%%%%%%%%%%%%%

\usepackage{algorithm}
\makeatletter
\renewcommand{\ALG@name}{Алгоритм}
\makeatother
\usepackage{algpseudocode}
\renewcommand{\algorithmicrequire}{\textbf{Вход:}}
\renewcommand{\algorithmicensure}{\textbf{Выход:}}

\usepackage{listings}   % code sections

%%%%%%%%%%%%%%%%%%%%%%%%%%%%%%%%%%%%%%%%%%%%%%%%%%%%%%%%%%%
%%%% Other
%%%%%%%%%%%%%%%%%%%%%%%%%%%%%%%%%%%%%%%%%%%%%%%%%%%%%%%%%%%

\usepackage{datetime}   % last modified time
\usepackage{wasysym}    % additional glyphs
\usepackage{verbatim}   % "\begin{verbatim}" -- for monospaced font
\usepackage{ifthen}     % \ifthenelse, \whiledo

\allowdisplaybreaks
