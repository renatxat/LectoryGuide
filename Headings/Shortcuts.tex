%%%%%%%%%%%%%%%%%%%%%%%%%%%%%%%%%%%%%%%%%%%%%%%%%%%%%%%%%%%
%%%% Handy information.
%%%%%%%%%%%%%%%%%%%%%%%%%%%%%%%%%%%%%%%%%%%%%%%%%%%%%%%%%%%

% def and newcommand difference: https://tex.stackexchange.com/questions/655/what-is-the-difference-between-def-and-newcommand

% ctrl + b -- \textbf{}
% ctrl + i -- \textit{}

%%%%%%%%%%%%%%%%%%%%%%%%%%%%%%%%%%%%%%%%%%%%%%%%%%%%%%%%%%%
%%%% Braces commands.
%%%%%%%%%%%%%%%%%%%%%%%%%%%%%%%%%%%%%%%%%%%%%%%%%%%%%%%%%%%

\newcommand{\pars}[1]{\left({#1}\right)}            % autosizable ()
\newcommand{\abs}[1]{\left|{#1}\right|}                 % autosizable ||
\newcommand{\norm}[1]{\left\|{#1}\right\|}              % autosizable \|\|
\newcommand{\set}[1]{\left\{{#1}\right\}}               % autosizable {}
\newcommand{\angles}[1]{\left\langle{#1}\right\rangle}    % autosizable <>
\newcommand{\condset}[2]{\set{{#1} \ | \ {#2}}}             % autosizable set definition
% \newcommand{\la}{\langle}
% \newcommand{\ra}{\rangle}

%%%%%%%%%%%%%%%%%%%%%%%%%%%%%%%%%%%%%%%%%%%%%%%%%%%%%%%%%%%
%%%% Calligraphyc letters (cal == calligraphyc).
%%%%%%%%%%%%%%%%%%%%%%%%%%%%%%%%%%%%%%%%%%%%%%%%%%%%%%%%%%%

\newcommand{\cA}{\mathcal{A}}
\newcommand{\cB}{\mathcal{B}}
\newcommand{\cC}{\mathcal{C}}
\newcommand{\cD}{\mathcal{D}}
\newcommand{\cG}{\mathcal{G}}
\newcommand{\cQ}{\mathcal{Q}}
\newcommand{\cR}{\mathcal{R}}
\newcommand{\cM}{\mathcal{M}}
\newcommand{\cN}{\mathcal{N}}
\newcommand{\cT}{\mathcal{T}}
\newcommand{\cP}{\mathcal{P}}
\newcommand{\cF}{\mathcal{F}}
\newcommand{\cX}{\mathcal{X}}
\newcommand{\cY}{\mathcal{Y}}

%%%%%%%%%%%%%%%%%%%%%%%%%%%%%%%%%%%%%%%%%%%%%%%%%%%%%%%%%%%
%%%% Letters for numeric sets (blackboard bold letter).
%%%%%%%%%%%%%%%%%%%%%%%%%%%%%%%%%%%%%%%%%%%%%%%%%%%%%%%%%%%

\renewcommand{\emptyset}{\varnothing}
\newcommand{\NN}{\mathbb{N}}
\newcommand{\ZZ}{\mathbb{Z}}
\newcommand{\QQ}{\mathbb{Q}}
\newcommand{\RR}{\mathbb{R}}
\newcommand{\CC}{\mathbb{C}}
\newcommand{\DD}{\mathbb{D}}
\newcommand{\EE}{\mathbb{E}}
\newcommand{\FF}{\mathbb{F}}
\newcommand{\II}{\mathbb{I}}
% \newcommand{\SS}{\mathbb{S}}

%%%%%%%%%%%%%%%%%%%%%%%%%%%%%%%%%%%%%%%%%%%%%%%%%%%%%%%%%%%
%%%% Greek letters.
%%%%%%%%%%%%%%%%%%%%%%%%%%%%%%%%%%%%%%%%%%%%%%%%%%%%%%%%%%%

\renewcommand{\phi}{\varphi}
\renewcommand{\kappa}{\varkappa}
\newcommand{\eps}{\varepsilon}
\newcommand{\lm}{\lambda}

%%%%%%%%%%%%%%%%%%%%%%%%%%%%%%%%%%%%%%%%%%%%%%%%%%%%%%%%%%%
%%%% Statements.
%%%%%%%%%%%%%%%%%%%%%%%%%%%%%%%%%%%%%%%%%%%%%%%%%%%%%%%%%%%

% \newcommand{\Paragraph}[1]{\section{Пункт {#1}}}
\newcommand{\Def}{\textbf{Def.} }
\newcommand{\Th}{\textbf{Теорема.} }
\newcommand{\Thbd}{\textbf{Теорема (б/д).} }
\newcommand{\Theor}[1]{\textbf{Теорема ({#1})}.}
\newcommand{\Theorbd}[1]{\textbf{Теорема ({#1}) (б/д)}.}
\newcommand{\Consequence}{\textbf{Следствие.} }
\newcommand{\Remind}{\textbf{Remind.} }
\newcommand{\Note}{\textbf{Note.} }
\newcommand{\Statement}{\textbf{Утверждение.} }
\newcommand{\Exercise}{\textbf{Упражнение.} }
\newcommand{\Prop}{\textbf{Свойство.} }
\newcommand{\Proof}{\textbf{Доказательство:} }
\newcommand{\Prooff}{\textbf{Доказать:} }
\newcommand{\Solution}{\textbf{Решение.} }
\newcommand{\Alg}{\textbf{Algorithm.} }
\newcommand{\Lemma}{\textbf{Лемма.} }
\newcommand{\Lemm}[1]{\textbf{Лемма ({#1})}.}
\newcommand{\Example}{\textbf{Пример:} }
\newcommand{\Task}{\textbf{Задача.} }
\newcommand{\Solve}{\textbf{Решение:} }
\newcommand{\Answer}{\textbf{Ответ:} }
\newcommand{\Examples}{\textbf{Примеры.} }
\newcommand{\Ex}{\textbf{Упражнение.} }
\newcommand{\Question}{\textbf{Вопрос:} }
\newcommand{\Sense}{\textbf{Смысл:} }
\newcommand{\Endproof}{$\hfill\blacksquare$ }

\let\bs\backslash
\let\tbs\textbackslash

\let\ra\rightarrow
\let\Ra\Rightarrow
\newcommand{\Lgra}{\quad\Longrightarrow\quad}

\let\la\leftarrow
\let\La\Leftarrow
\newcommand{\Lgla}{\quad\Longleftarrow\quad}

\let\lra\leftrightarrow
\let\Lra\Leftrightarrow
\newcommand{\Lglra}{\quad\Longleftrightarrow\quad}

\let\rra\rightrightarrows
\let\hra\hookrightarrow

\let\wdt\widetilde
\let\wdh\widehat

%%%%%%%%%%%%%%%%%%%%%%%%%%%%%%%%%%%%%%%%%%%%%%%%%%%%%%%%%%%
%%%% Inequalities (Russian style).
%%%%%%%%%%%%%%%%%%%%%%%%%%%%%%%%%%%%%%%%%%%%%%%%%%%%%%%%%%%

\renewcommand{\le}{\leqslant}
\renewcommand{\ge}{\geqslant}

%%%%%%%%%%%%%%%%%%%%%%%%%%%%%%%%%%%%%%%%%%%%%%%%%%%%%%%%%%%
%---------------------------------------------------------%
%%%%%%%%%%%%%%%%%%%%%%%%%%%%%%%%%%%%%%%%%%%%%%%%%%%%%%%%%%%

%%%%%%%%%%%%%%%%%%%%%%%%%%%%%%%%%%%%%%%%%%%%%%%%%%%%%%%%%%%
%%%% Common.
%%%%%%%%%%%%%%%%%%%%%%%%%%%%%%%%%%%%%%%%%%%%%%%%%%%%%%%%%%%

\newcommand{\deriv}[2]{\frac{\partial {#1}}{\partial {#2}}}

\DeclareMathOperator*{\argmax}{arg\,max} % * allows display limits under argmax
\DeclareMathOperator*{\argmin}{arg\,min}

\newcommand{\sumlim}[2]{\sum\limits_{{#1}}^{{#2}}}
\newcommand{\prodlim}[2]{\prod\limits_{{#1}}^{{#2}}}

%%%%%%%%%%%%%%%%%%%%%%%%%%%%%%%%%%%%%%%%%%%%%%%%%%%%%%%%%%%
%%%% Mathematical statistics.
%%%%%%%%%%%%%%%%%%%%%%%%%%%%%%%%%%%%%%%%%%%%%%%%%%%%%%%%%%%

\DeclareMathOperator{\E}{\mathbb{E}}
\DeclareMathOperator{\D}{\mathbb{D}}
\newcommand{\Xmean}{\overline{X}}
\newcommand{\Ymean}{\overline{Y}}
\newcommand{\samp}[2]{{#1}_1, \ldots, {#1}_{#2}}
\newcommand{\observation}{X = (\samp{X}{n})}
\newcommand{\distrfamily}{\condset{P_\theta}{\theta \in \Theta}}

%%%%%%%%%%%%%%%%%%%%%%%%%%%%%%%%%%%%%%%%%%%%%%%%%%%%%%%%%%%
%%%% Linear algebra.
%%%%%%%%%%%%%%%%%%%%%%%%%%%%%%%%%%%%%%%%%%%%%%%%%%%%%%%%%%%

\DeclareMathOperator{\re}{Re}   % Real part of complex value.
\DeclareMathOperator{\im}{Im}   % Imaginary part of complex value.
\DeclareMathOperator{\imm}{Im}  % Image of a function.
\DeclareMathOperator{\kerr}{Ker}
\DeclareMathOperator{\dom}{dom}
\DeclareMathOperator{\tr}{Tr}
\DeclareMathOperator{\dist}{dist}
\DeclareMathOperator{\diag}{diag}
\DeclareMathOperator{\proj}{proj}
\DeclareMathOperator{\sign}{sign}
\DeclareMathOperator{\rank}{rank}

%%%%%%%%%%%%%%%%%%%%%%%%%%%%%%%%%%%%%%%%%%%%%%%%%%%%%%%%%%%
%%%% Calculations complexity.
%%%%%%%%%%%%%%%%%%%%%%%%%%%%%%%%%%%%%%%%%%%%%%%%%%%%%%%%%%%

% \mathsf{} - name of class

%%%%%%%%%%%%%%%%%%%%%%%%%%%%%%%%%%%%%%%%%%%%%%%%%%%%%%%%%%%
%%%% Matrices.
%%%%%%%%%%%%%%%%%%%%%%%%%%%%%%%%%%%%%%%%%%%%%%%%%%%%%%%%%%%

% Блочная матрица
\newcommand{\blockmatrix}[9]{
  \draw[draw=#4,fill=#5] (0,0) rectangle( #1,#2);
  \ifthenelse{\equal{#6}{true}}
  {
    \draw[draw=#7,fill=#8] (0,#2) -- (#9,#2) -- ( #1,#9) -- ( #1,0) -- ( #1 - #9,0) -- (0,#2 -#9) -- cycle;
  }
  {}
  \draw ( #1/2, #2/2) node { #3};
}

% Правая скобка произвольного размера
% \rightparen{width}
\newcommand{\rightparen}[1]{
  \begin{tikzpicture}
    \draw (0,#1/2) arc (0:30:#1);
    \draw (0,#1/2) arc (0:-30:#1);
  \end{tikzpicture}%this comment is necessary
}

% Левая скобка произвольного размера
% \leftparen{width}
\newcommand{\leftparen}[1]{
  \begin{tikzpicture}
    \draw (0,#1/2) arc (180:150:#1);
    \draw (0,#1/2) arc (180:210:#1);
  \end{tikzpicture}%this comment is necessary
}

% Unframed block matrix, "m" prefix to match fbox, mbox
% \blockmatrix[r,g,b]{width}{height}{text}
\newcommand{\mblockmatrix}[4][none]{
  \begin{tikzpicture}
    \ifthenelse{\equal{#1}{none}}
    {
      \blockmatrix{#2}{#3}{#4}{none}{none}{false}{none}{none}{0.0}
    }
    {
      \definecolor{fillcolor}{rgb}{#1}
      \blockmatrix{#2}{#3}{#4}{none}{fillcolor}{false}{none}{none}{0.0}
    }
  \end{tikzpicture}%this comment is necessary
}

% Framed block matrix
% \fblockmatrix[r,g,b]{width}{height}{text}
\newcommand{\fblockmatrix}[4][none]{
  \begin{tikzpicture}
    \ifthenelse{\equal{#1}{none}}
    {
      \blockmatrix{#2}{#3}{#4}{black}{none}{false}{none}{none}{0.0}
    }
    {
      \definecolor{fillcolor}{rgb}{#1}
      \blockmatrix{#2}{#3}{#4}{black}{fillcolor}{false}{none}{none}{0.0}
    }
  \end{tikzpicture}%this comment is necessary
}

% Diagonal block matrix
% \dblockmatrix[r,g,b]{width}{height}{text}
\newcommand{\dblockmatrix}[4][none]{
  \begin{tikzpicture}
    \ifthenelse{\equal{#1}{none}}
    {
      \blockmatrix{#2}{#3}{#4}{black}{none}{true}{black}{none}{0.35cm}
    }
    {
      \definecolor{fillcolor}{rgb}{#1}
      \blockmatrix{#2}{#3}{#4}{black}{none}{true}{black}{fillcolor}{0.35cm}
    }
  \end{tikzpicture}%this comment is necessary
}


% Diagonal block matrix, but exposes diagonal offset
% \diagonalblockmatrix[r,g,b]{width}{height}{text}
\newcommand{\diagonalblockmatrix}[5][none]{
  \begin{tikzpicture}

    \ifthenelse{\equal{#1}{none}}
    {
      \blockmatrix{#2}{#3}{#4}{black}{none}{true}{black}{none}{#5}
    }
    {
      \definecolor{fillcolor}{rgb}{#1}
      \blockmatrix{#2}{#3}{#4}{black}{none}{true}{black}{fillcolor}{#5}
    }

  \end{tikzpicture}%necessary comment
}

\newcommand{\valignbox}[1]{
  \vtop{\null\hbox{#1}}% necessary comment
}

% a hack so that I don't have to worry about the number of columns or
% spaces between columns in the tabular environment
\newenvironment{blockmatrixtabular}
{% necessary comment
  \begin{tabular}{
    @{}l@{}l@{}l@{}l@{}l@{}l@{}l@{}l@{}l@{}l@{}l@{}l@{}l@{}l@{}l@{}l@{}l@{}l@{}l
    @{}l@{}l@{}l@{}l@{}l@{}l@{}l@{}l@{}l@{}l@{}l@{}l@{}l@{}l@{}l@{}l@{}l@{}l@{}l
    @{}l@{}l@{}l@{}l@{}l@{}l@{}l@{}l@{}l@{}l@{}l@{}l@{}l@{}l@{}l@{}l@{}l@{}l@{}l
    @{}
    }
    }
    {
  \end{tabular}%necessary comment
}

%%%%%%%%%%%%%%%%%%%%%%%%%%%%%%%%%%%%%%%%%%%%%%%%%%%%%%%%%%%
%%%% Other
%%%%%%%%%%%%%%%%%%%%%%%%%%%%%%%%%%%%%%%%%%%%%%%%%%%%%%%%%%%

\newcommand{\pmitem}[1]{<<\textsf{{#1}}>>}   % to emphasize menu items
\newcommand{\bmitem}[1]{\textbf{\textsf{{#1}}}}   % to emphasize menu items
\newcommand{\mitem}[1]{\textsf{{#1}}}   % to emphasize menu items
\newcommand{\textbfit}[1]{\textbf{\textit{{#1}}}}

\newcommand{\caret}{$^\wedge$} % circumflex/hat
