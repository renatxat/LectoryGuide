\section{Для новобранцев}\label{sec:for-newcomers}

%%%%%%%%%%%%%%%%%%%%%%%%%%%%%%%%%%%%%%%%%%%%%%%%%%%%%%%%%
\subsection{Общее}
%%%%%%%%%%%%%%%%%%%%%%%%%%%%%%%%%%%%%%%%%%%%%%%%%%%%%%%%%

Предположим, ты недавно оказался в составе команды Лектория ФПМИ. Работа здесь разделяется на следующие роли:

\begin{itemize}
  \item Оператор
  \item Монтажёр
  \item Ответственный за аппаратуру
  \item Ответственный за техническую часть
  \item Заместители главы лектория
  \item Глава лектория
\end{itemize}

В начале семестра происходит распределение, кто какой предмет будет снимать и монтировать. Для этого достаточно записаться на предпочтительные предметы в табличке <<Запись на предметы>>, лежащей на диске (смотри \hyperref[sec:cloud-storage]{главу про облачное хранилище}). Перед самым началом семестра глава лектория или его заместители связываются с преподавателями и просят разрешения на запись их лекций в этом семестре. Если всё ок, то это помечается в этой же табличке. Во время семестра после записи/монтажа лекции нужно отметить это в таблице <<Учёт лекций>>, которая также лежит на диске (обычно это делает монтажёр).

У ответственных за оборудование хранится оборудование, которое принадлежит лекторию (формально --- студсовету). В частности, они отвечают за переносные камеры: отдают их днём на съёмки, вечером заряжают и загружают записи на диск, отвечают за автоматику (несколько скриптов), которая обрабатывает записи с камер для удобства монтажёров.

Для начала стоит прочитать \hyperref[sec:infrastructure]{главу про инфраструктуру} лектория, после чего изучить отдельные главы в зависимости от вашей роли.

%%%%%%%%%%%%%%%%%%%%%%%%%%%%%%%%%%%%%%%%%%%%%%%%%%%%%%%%%
\subsection{Обязанности операторов}
%%%%%%%%%%%%%%%%%%%%%%%%%%%%%%%%%%%%%%%%%%%%%%%%%%%%%%%%%

\textbf{Операторы} обязательно посещают все лекции по предмету либо ищут человека, который сможет их заменить на конкретной лекции. Перед началом семестра оператор должен узнать, стоит ли в аудитории институтская камера (наличие смотри в \hyperlink{main-chat-tg}{основном чате}). Инструкция по тому, как снимать на институтскую установку, расположена в \iffalse \hyperref[sec:institution-camera]{главе про институтские установки}\fi (TODO: написать инструкцию). Если камеры в аудитории нет в этом семестре либо оказалось, что она не работает (или её скоро уберут), нужно обратиться к кому-то из ответственных за оборудование (смотри \hyperref[sec:portable-camera]{главу о переносной камере}) и договориться с ними о переносной камере.


%Перед началом семестра оператор должен узнать, стоит ли в аудитории институтская камера (наличие смотри в \hyperref[ssec:equipment-availability]{разделе про наличие установок} либо спрашивай в \hyperlink{main-chat-tg}{основном чате}). Инструкция по тому, как снимать на институтскую установку, расположена в \hyperref[sec:institution-camera]{главе про институтские установки}. Если камеры в аудитории нет в этом семестре либо оказалось, что она не работает (или её скоро уберут), нужно обратиться к кому-то из ответственных за оборудование (смотри \hyperref[sec:portable-camera]{главу о переносной камере}) и договориться с ними о переносной камере.

Иногда перед началом лекции возникают проблемы: не включается камера, нет звука, ... В таких случаях быструю помощь можно получить в чате \hyperlink{emergency-chat-tg}{emergency} (его лучше не мьютить). Перед обращением туда, пробегитесь глазами по этому мануалу и убедитесь, что в нём не описано решение вашей проблемы.

%%%%%%%%%%%%%%%%%%%%%%%%%%%%%%%%%%%%%%%%%%%%%%%%%%%%%%%%%
\subsection{Обязанности монтажёров}
%%%%%%%%%%%%%%%%%%%%%%%%%%%%%%%%%%%%%%%%%%%%%%%%%%%%%%%%%

Про обязанности \textbf{монтажёра} можно посмотреть в \hyperlink{montage-guide}{путеводителе по монтажу}. Ссылка на него, а также много других полезных ссылок расположены в разделе \hyperref[ssec:important-links]{важные ссылки}.
