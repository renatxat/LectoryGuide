\section{Институтские камеры (установки)}\label{sec:institution-camera}

%%%%%%%%%%%%%%%%%%%%%%%%%%%%%%%%%%%%%%%%%%%%%%%%%%%%%%%%%
\subsection{Наличие аппаратуры в аудиториях (осень 2024)}\label{ssec:equipment-availability}
%%%%%%%%%%%%%%%%%%%%%%%%%%%%%%%%%%%%%%%%%%%%%%%%%%%%%%%%%

Состояние ОК означает, что в аудитории корректно работает камера, есть 2 набора батареек и есть возможность записывать проектор через OBS без подключения интернета, а также есть доступ в интернет и подключён сетевой диск.

\begin{itemize}
  \item \textbf{115 КПМ}: ОК.

  \item \textbf{202 НК}: Оборудование на месте, но затестить работоспособность не получилось.

  \item \textbf{239 НК}: Оборудование есть, обс не видит камеру (аверку причем видит, но показывает черный экран).

  \item \textbf{113 ГК}:
        \begin{enumerate}[label=\alph*), noitemsep]
          \item Малый фпс на камере.
          \item Не захватывается презентация. Возможное решение: поменять разрешение сторон источника.
        \end{enumerate}

  \item \textbf{Поточка Арктики}:
        \begin{enumerate}[label=\alph*), noitemsep]
          \item Нет hdmi от проектора --- проектор нужно снимать через зум.
        \end{enumerate}

  \item \textbf{Поточка Цифры}: ОК.

  \item \textbf{БХим}: ОК.

  \item \textbf{БФиз}: ОК.

  \item \textbf{230 ГК}: Камера, монитор, аверки, комп на месте, но нет удлинителя (без него записывать не получится).
\end{itemize}

%%%%%%%%%%%%%%%%%%%%%%%%%%%%%%%%%%%%%%%%%%%%%%%%%%%%%%%%%
\subsection{Всё остальное}
%%%%%%%%%%%%%%%%%%%%%%%%%%%%%%%%%%%%%%%%%%%%%%%%%%%%%%%%%

\partodo{
  Написть про различные батарейки в аудиториях.

  На макбуке разрешение экрана не 16:9, поэтому при подключении его к проектору в обс может не быть картинки и светиться надпись "adjust resolution".

  Удаление записей на компах. Как на аккаунте лектория, так и на операторском.
}
